
% Tìm hiểu thêm về các biến thể ví dụ như đã có một vài ô quân hậu được đặt trước

\chapter{Kết luận}
Chúng ta đã so sánh bài toán xếp hậu với các phương pháp cổ điển về mặt thời gian chạy. Việc tìm ra lời giải chính xác cho các bài toán tối ưu hóa lớn, phức tạp bằng các phương pháp cổ điển về cơ bản là một ngõ cụt tính toán. Không gian lời giải cho một bài toán tối ưu hóa lớn nhanh chóng vượt quá tầm kiểm soát và việc tìm kiếm trong tất cả các lời giải khả thi trở nên khó khăn.

Công thức đề xuất của chúng ta, công thức cộng dồn, tuy chưa đem đến kết quả đáng mong đợi nhưng về khía cạnh tối ưu hóa số lượng liên kết giữa các biến và chỉ yêu cầu lời giải gần đúng vẫn có thể coi là chấp nhận được.

Chúng ta cũng đã thử nghiệm và so sánh bộ lấy mẫu D-Wave với ủ mô phỏng và nhận thấy rằng thời gian chạy của nó trong bài toán xếp hậu được đánh giá không tốt. Xếp hậu là một bài toán dẫn xuất từ bài toán bao phủ chính xác, tuy nhiên ràng buộc của nó quá dày, tương tác hạt lượng tử nhiều, đặc biệt cấu trúc hạt lượng tử thưa như Pegasus sẽ yêu cầu sử dụng nhiều hạt lượng tử đã dẫn đến kết quả ủ lượng tử tệ hơn ủ mô phỏng. Hy vọng rằng cải tiến phần cứng có thể thay đổi cũng như cải thiện, khắc phục hạn chế này.

Độ phức tạp về thời gian có tác động lớn đối với quá trình ủ mô phỏng và nó không phải là hằng số. Ngược lại, quá trình ủ lượng tử thực sự có độ phức tạp về thời gian không đổi, là một điểm sáng đầy hứa hẹn.

Ủ mô phỏng là xác suất; về nguyên tắc, chúng ta có thể quyết định sự cân bằng giữa tốc độ và độ chính xác.
Trong tương lai, nếu không gian bài toán có các đặc điểm đã biết ví dụ như cho trước một vài quân hậu thì chúng ta có thể đưa ra các giả định đơn giản hóa có thể được khai thác tốt hơn. Nếu không gian bài toán của chúng ta hoàn toàn chưa được khám phá, thì tính chính xác của lời giải sẽ phụ thuộc vào tính ngẫu nhiên của không gian bài toán, thời gian chờ đợi và may mắn. Tùy theo yêu cầu cho bài toán khác nhau, việc ủ mô phỏng có thể đã đủ tốt, có khi là vượt trội.

